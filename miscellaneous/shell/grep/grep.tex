\documentclass{article}
\usepackage{xeCJK}
\usepackage{amsmath}
\usepackage{listings}
\usepackage{xcolor}
\setlength{\parindent}{0pt}
\renewcommand{\baselinestretch}{1.0}
\lstset{
	frame=tb, % draw a frame at the top and bottom of the code block
	showstringspaces=false, % don't mark spaces in strings
	numbers=left, % display line numbers on the left
	commentstyle=\color{green}, % comment color
	keywordstyle=\color{blue}, % keyword color
	stringstyle=\color{red} % string color
}
\usepackage[a4paper,left=20mm,right=20mm,top=15mm,bottom=15mm]{geometry}  

\title{grep}
\author{MengChunlei}

\begin{document}
\maketitle
\section{基本用法}
\begin{lstlisting}[language=bash]
mcl@mcl: cat file1 
bhishek 20 cc 11
chitransh 102 vv 20
zzz 8 zz 10
zzz 8 yy 7
zzz 8 bb 20
xyz 77 aa 12
satish 103 ff 100
satish 103 ff 1
mcl@mcl: grep zzz file1  /*查找含有zzz的行*/
zzz 8 zz 10
zzz 8 yy 7
zzz 8 bb 20
mcl@mcl: grep -n zzz file1 /*输出中带有行号*/
3:zzz 8 zz 10
4:zzz 8 yy 7
5:zzz 8 bb 20
mcl@mcl: grep -v zzz file1 /*查找不含有zzz的行*/
bhishek 20 cc 11
chitransh 102 vv 20
xyz 77 aa 12
satish 103 ff 100
satish 103 ff 1
mcl@mcl: grep -c zzz file1  /*统计行数*/
3
mcl@mcl: cat file2 
my name is mcl
haha
My NAME is mcl
mcl@mcl: grep -i name file2  /*忽略大小写*/
my name is mcl
My NAME is mcl
mcl@mcl: grep -q zzz file1 /*-q不输出,但是可以利用其返回值进行条件判断*/
mcl@mcl: echo $?
0
mcl@mcl: grep -q xxxx file1 
mcl@mcl: echo $?
1

\end{lstlisting}

\section{环顾功能}
\begin{lstlisting}[language=bash]
mcl@mcl: cat file3 
start
first LINE
second line
third LINE
last line
END
mcl@mcl: grep -A 1 LINE file3 /*输出匹配行的后面一行*/
first LINE
second line
third LINE
last line
mcl@mcl: grep -B 1 LINE file3 /*输出匹配行的前面一行*/
start
first LINE
second line
third LINE
mcl@mcl: grep -C 1 LINE file3 /*输出匹配行的前后各一行*/
start
first LINE
second line
third LINE
last line
\end{lstlisting}

\section{多文件}
\begin{lstlisting}[language=bash]
mcl@mcl: ls
file1  file2  file3
mcl@mcl: grep -l name file*  /*输出含有name的文件*/
file2
mcl@mcl: grep -L name file* /*输出不含有name的文件*/
file1
file3
mcl@mcl: echo "aaa" > file1
mcl@mcl: echo "bbb" > file2
mcl@mcl: echo "aaa" > file3
mcl@mcl: ls
file1  file2  file3
mcl@mcl: grep "aaa" file* -lZ 
file1file3mcl@mcl: 
mcl@mcl: 
mcl@mcl: grep "aaa" file* -lZ | xargs -0 /*可以将这些文件列出在一行*/
file1 file3
mcl@mcl: grep "aaa" file* -lZ | xargs -0 rm 
mcl@mcl: ls
file2

\end{lstlisting}

\section{正则表达式}
\begin{lstlisting}[language=bash]
mcl@mcl: cat file4
a test
test it
last test again
mcl@mcl: grep '^test' file4 /*查找有test开头的行*/
test it
mcl@mcl: grep 'test$' file4 /*查找有test结尾的行*/
a test
mcl@mcl: grep 'test$|^test' file4
mcl@mcl: grep -E 'test$|^test' file4 /*查找有test开头或者结尾的行*/
a test
test it
mcl@mcl: egrep 'test$|^test' file4 /*查找有test开头或者结尾的行,可以直接用egrep*/
a test
test it
\end{lstlisting}

\section{按照单词查找}
\begin{lstlisting}[language=bash]
mcl@mcl: cat file5
i love you
it glove it
mcl@mcl: grep love file5
i love you
it glove it
mcl@mcl: grep -w love file5 /*w选项可以只查找单词*/
i love you
\end{lstlisting}

\section{目录查找}
\begin{lstlisting}[language=bash]
mcl@mcl: tree project/
project/
├── a.h
├── b.h
├── common
│   └── c.h
└── README.txt

1 directory, 4 files
mcl@mcl: cat project/a.h 
void print(int x);
mcl@mcl: cat project/b.h 
int add(int x, int y) {
  return x + y;
}
void print(double t);
mcl@mcl: cat project/README.txt 
print
add
mcl@mcl: cat project/common/c.h 
void print(char c);
mcl@mcl: grep -r print project/  /*在project下面的所有文件中查找print*/
project/README.txt:print
project/common/c.h:void print(char c);
project/a.h:void print(int x);
project/b.h:void print(double t);
mcl@mcl: grep -r print project --include=*.h /*只在.h文件中查找*/
project/common/c.h:void print(char c);
project/a.h:void print(int x);
project/b.h:void print(double t);
mcl@mcl: grep -r print project --include=*.txt /*只在.txt文件中查找*/
project/README.txt:print
mcl@mcl: grep -r print project --include=*.{txt,h} /*在.txt或者.h文件中查找*/
project/README.txt:print
project/common/c.h:void print(char c);
project/a.h:void print(int x);
project/b.h:void print(double t);
mcl@mcl: grep -r print project --exclude=*.txt /*排除.txt文件*/
project/common/c.h:void print(char c);
project/a.h:void print(int x);
project/b.h:void print(double t);
mcl@mcl: grep -r print project --exclude-dir=common/ /*排除common文件夹*/
project/README.txt:print
project/a.h:void print(int x);
project/b.h:void print(double t);
\end{lstlisting}

\end{document}
